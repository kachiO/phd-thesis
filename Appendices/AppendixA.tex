% Appendix A

\chapter{Materials and Methods} % Main chapter title

\label{AppendixA} % Change X to a consecutive number; for referencing this chapter elsewhere, use \ref{ChapterX}

\section{Animal Subjects}
All animal procedures and experiments were approved by the Cold Spring Harbor Laboratory Animal Care and Use Committee. Experiments were conducted with female or male mice between the ages of 6-25 weeks. Mouse strains were either C57BL/6J or SV129 background. Trangenic mice used included Ai93/Emx-tTA/CamKII $\alpha$-tTA (GCaMP6f), Ai95/Emx-cre (GCaMP6f), Gad2-Cre, and PV-IRES-Cre. \\
For behavioral performance comparison, male Long Evan rats (6 weeks, Taconic) were also used. 

\section{Headbar Implantation and Skull Preparation}
For retinotopic mapping measurements, mice were implanted with a custom titanium headbar. Mice were anesthetized with isoflurane (2\%) mixed with oxygen and secured onto a stereotaxic apparatus. Body temperature was maintained at 37 $^{\circ}$ C with a rectal temperature probe. The eyes were lubricated with eye ointment before the start of the surgery, followed by subcutaneous injection of analgesia (Meloxicam, 2mg/kg) and antibiotic (Enrofloxacin, 2mg/kg). Fur on the scalp was removed with hair clippers and Nair (Sensitive Formula with Green Tea), followed by betadine (5\%) swab. Lidocaine (100 $\mu $L) was injected underneath the scalp before removing the scalp. The skull was cleaned with saline and allowed to dry. A generous amount of vetbond tissue glue (3M) was then applied to seal the skull. Once the vetbond was dry, the headbar was secured with Metabond (Parkwell) and dental acrylic. Mice were allowed 3 days to recover before retinotopic mapping. 
 
\section{Window Implantation} 
Window implantation is optional for retintoptic mapping, as the mouse skull is translucent. This procedure requires sterile technique and is typically done in conjunction with the headbar implant. The optical window implantation procedure was adopted from \textcite{Goldey2014}. Briefly, the mouse was anesthetized with isoflurane and secured onto a stereotaxic instrument (Kopf). The eyes were lubricated with eye ointment before the start of the surgery, and the mouse was administered subcutaneous injections of Dexamethasone (4 mg/kg) to prevent swelling of the brain, Meloxicam (2mg/kg), and Enrofloxacin (2mg/kg). The scalp and skull was prepared using the same technique described above. Vetbond tissue glue was applied to skull, followed by a unilateral craniotomy using a disposable biopsy punch (Milltex) of the desired size (4 mm). Three round glass coverslips (2-4mm and 1-5mm, Warner Scientific) glued together with optical glue (Norland Optical Adhesives, \#71) were used to seal the craniotomy with Vetbond. Post-operative analagesia (Meloxicam, 2mg/kg) was administered daily for 5 days following the surgery.  Mice were allowed 5-10 days to recover before retinotopic mapping.  

\section{Retinotopic Mapping}
Retinotopic mapping was performed in awake headfixed animals using one of two procedures, episodic \parencite{Gias2005,Andermann2011} or periodic (Fourier) stimulation \parencite{Kalatsky2003,Garrett2014,Juavinett2016}. For episodic stimulation, spatially restricted drifting gratings (25-40$^{\circ}$ in diameter) were presented to the hemifield contralateral to the window implant or exposed skull. A typical trial proceeded in the following sequence: 4s blank, 4s stimulus, and 4s blank. Each trial was followed by a 6 to 8s inter trial interval. A gray screen was presented during blank periods, and the the stimulus was presented on a gray background. Each stimulus location was repeated 20-80 times. \\
For periodic (Fourier) stimulation, a narrow bar (5-10 $^{\circ}$) was drifted across the four cardinal directions of the screen. Presented within the drifting bar was a flickering checkerboard pattern (12 $^{\circ}$ checks, 5Hz). One trial consisted of 11 sweeps of the bar in 22 secs in one of the four cardinal directoins, however the the first cycle was discarded because it typically introduced stimulus onset transients. Each trial was repeated 15 times for each direction. The monitor was placed in contralateral visual hemifield to imaging hemisphere, positioned at an angle of 77 $^{\circ}$ from the midline of the mouse and a distance of 15 cm.The procedures for periodic retinotopic mapping were adopted from \parencite{Juavinett2016}. Imaging data was acquired at frame rates between 5 to 50 fps. 

\section{Behavioral Training}
Before behavioral training, mice were gradually water restricted over the course of a week. Mice were weighed daily and checked for signs of dehydration throughout training period. Mice that weighed less than 80\% of their original pre-training weight were supplemented with additional water. Behavioral training sessions typically lasted 1-2 hours, daily, in which mice typically harvested at least 1 mL of water. Mice rested on the weekends. If a mouse failed to harvest at least 0.4 mL on two consecutive days, the mouse was supplemented with additional water.\par
Animal training took place in sound isolation chamber, which contained a three-choice port box. The mouse would poke into the center port to initiate trials and deliver the stimulus. Given the stimulus, the animal reported its choice on either the left- or right-side port. In the first training stage, mice learned to wait for at least 1100 ms at the center port before reporting their decision. This behavior was shaped by rewarding the mice at the center port (0.5 $\mu $L) and gradually increasing the minimum wait duration from 25 ms to 1100 ms over the course of 1-2 behavioral sessions. Without center reward, this stage typically took 10-12 sessions to learn. During the first stage of training, the stimulus is played and the reward (2 $\mu$L) is automatically dispensed at either the left or right side depending on the stimulus. In this phase the mouse is not punished for an incorrect first choice.  In the second stage of training, the mouse learns to discriminate the stimulus and is required to make the correct choice on the first attempt. Initially, mice are biased during this stage and several anti-bias methods are employed to correct the bias, such as physically obstructing access the biased port, changing the reward size, or proportion of left vs right trials. The mouse is considered trained once they are unbiased, performing above chance, and experiencing all stimulus strengths.\par
Trial type, stimulus and reward delivery, control, and data collection was performed through a MATLAB interface and Arduino-powered device called BPod (SanWorks LLC).\par 

\section{Stimulus Generation}
The stimulus consists of a sequence of 20 ms pulses of light from a LED panel (Ala Scientific). The inter-pulse intervals are randomized from a uniform or exponential distribution. In earlier experiments (fixed interval), the intervals were drawn from a uniform distribution of 50 or 100 ms, and each pulse was 15 ms. In the fixed interval stimulus, the event rates were between 9 to 16 flashes/s with a category boundary at 12.5 flashes/s. For the exponential interval stimulus, the minimum inter-pulse interval was 20 ms, and the maximum interval was determined by the number of flashes for a given stimulus. The number of flashes were between 4-20 flashes/s. The stimulus was created using 25 fixed time bins, each 20ms in duration. A Poisson coin was flipped to determine whether an event (flash) would occur in each bin. Each fixed time bin was followed by an empty 20ms time bin. 
\subsection{Brightness Manipulation}
Each 20ms flash pulse was generated by a half-wave rectified sinusoidal signal thresholded at the peaks and with a base frequency of 200Hz. This approach effectively controls the total LED on-time or the "density" of the 20ms pulses. It is similar to pulse-width modulation technique used to control LED brightness. During normal sessions, the base frequency is multiplied by a brightness factor, which is kept constant across sessions. In the first brightness manipulation experiment, the normal brightness factor was either halved or doubled on 5\% of trials to produce the "dimmer" and "brighter" conditions. In the second brightness manipulation experiment, the normal brightness factor was inversely scaled with the flash rate such that the brightness factor was normalized by the quotient of the flash rate of the current trial and the lowest flash rate (4 flashes/s). This manipulation was introduced on 5\% of trials.  

\section{Chemogenetics Experiments}
Mice (Gad2-IRES-cre) were injected with AAV8-hSyn-DIO-hm4Di-mCherry into stereotactic mouse PPC \parencite{Harvey2012}. Approximately 200nL of virus was injected in both hemispheres at 200 $\mu$m below the surface of the brain.  The mice were allowed to 3 days to recover before behavioral training. \\
CNO (Sigma or Enzo Life Sciences) was diluted to a final concentration of 0.5mg/ml in saline and DMSO, and stored in aliquots of 0.4ml in the -80 $^\circ$ C. Each CNO aliquot was used only once after thawing, Mice were administered a dosage of 2mg/kg (or 100 $\mu$ L/25g mouse) of CNO or an equivalent volume of saline (0.9\%). Care was taken to ensure that there was at least two-three days between CNO treatments to allow full recovery and stabilization of behavior. 

\section{Survival Blood Collection via Lateral Tail Vein (Mouse)}
Mouse was placed on warm heating pad. The temperature should not exceed 85-90 $^\circ$ F. The mouse was then placed in a plastic restrainer.The tail was cleaned with 70\% ethanol. The lateral tail vein was nicked With a \#11 scalpel blade. Blood(15-30 $\mu$L was collected into a receptacle. Excess blood was cleaned up with Kimwipe and the wound sealed with Vetbond. The blood sample was centrifuged, and the blood plasma extracted. The plasma was submitted to and analyzed by the CSHL Proteomics Facility. 

\section{Optogenetic Inactivation}
For Jaws inhibition experiments, mice were injected with AAV8-hSyn-Jaws-KGC-GFP-ER2 (Group 1) or AAV8-CamKII-Jaws-KGC-GFP-ER2 (Group 2) into area AM identified by retinotopic mapping. Virus injections were performed using Drummond Nanoject III, which enables automated delivery of small volumes of virus. To minimize virus spread, the Nanoject was programmed to inject slowly: six 30 nL boluses, 60 s apart, and each bolus delivered at 10 nL/sec.
Approximately 180nL of virus was injected at multiple depths ($\sim$200 and $\sim$500 $\mu $m) below the brain surface. Following the virus injection, 200 $\mu$m fiber (metal ferrule, ThorLabs) was implanted above the injection site. The optical fiber was secured onto the skull with vitrebond, metabond, and dental acrylic. The animals was allowed at least 3 days to recover before behavioral training. \\
A red 640nm fiber-coupled laser (OptoEngine) was used for inactivation. Experiments were conducted with multiple laser power levels: 0.5, 1, and 2 mW. One power level was used per session. On inactivation sessions, laser light was externally triggered using a PulsePal (Sanworks LLC) device. The laser stimulation pattern was a square pulse (1 second long) followed by a linear ramp (0.25s), which began at the onset of the stimulus. Stimulation occurred on 25\% of trials. 




