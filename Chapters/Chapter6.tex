% Chapter Template

\chapter{Conclusion and Perspectives} % Main chapter title

\label{Chapter6} 

In this thesis, I have demonstrated a quantitative behavioral paradigm for studying temporal accumulation of visual evidence in mice. Mice have been largely underrepresented in studies investigating temporal sensory evidence accumulation, and until recently it remained unclear whether mice could be reliably trained to perform perceptual decision making tasks that require accumulation of sensory evidence over time. Results from cortical manipulation of mouse PPC defined by stereotactic coordinates demonstrated that the accurate performance on the task was depended on cortex, and in particular on normal activity of inhibitory neurons. Furthermore, targeted photoinhibition of secondary visual area AM, located in the mouse parietal cortex area, putatively contributes to evidence accumulation behavior. More precisely, it appears to be involved in guiding contralateral movements. More work is needed to resolve the role of visual area AM as the results are confounded by the artifact of \emph{in vivo} red light stimulation. Finally, I found that population activity of mouse V1 can linearly distinguish between different natural scene images, at a much greater accuracy than other higher order visual areas.\par

Overall, I demonstrated for the first time that a secondary visual area in the mouse contributes to decision making tasks that require evidence accumulation. These experiments lay groundwork for further investigations of the neural circuits that underlie evidence accumulation of visual evidence across time in mouse. Such experiments should focus on using the tools available in mouse to characterize the nature of the neural signals underlying evidence accumulation as well as how this signal influences downstream processes to direct behavior. Below, I outline in detail some of the experiments necessary to address these questions.\par 

The first avenue for future efforts is to develop a computational model of sensory evidence accumulation that quantitatively characterizes the psychophysical behavior of the mice, taking into account the observations identified in Chapter \ref{Chapter2}. A good place to begin would be with the extended drift diffusion accumulator model developed by \parencite{Brunton2013}. The Brunton model uses trial-by-trial stimulus information to estimate the noise associated with the evidence accumulator. Among the main features of the model is that estimates the accumulator noise separately from the noise associated with the incoming sensory evidence. Included in the model are parameters that account for sensory adaptation, which would be useful in determining to what extent subjects use a brightness in solving the task. The model could also be used to determine the optimal shape of the psychophysical kernel, and whether mice are accumulating evidence optimally. More interestingly, the computational model could generalize to multiple evidence accumulation tasks that use pulsatile sequences. Currently there is not a consensus on sensory evidence accumulation tasks and it is unclear whether the brain treats has a generalized solution for solving such tasks. For instance the task developed in this thesis utilizes a single stream of evidence that is non-spatial, however, several paradigms have also used two independent spatially-localized streams  \parencite{Sanders2012,Brunton2013,Marcos2016,Katz2016}. With a general computational framework, it would be interesting to compare the different modes (one vs two stream) of evidence accumulation and different different species trained on the same task. \par 

An open avenue to explore is the role of the corticostriatal connection, namely the prominent projection from area AM to striatum. Although the striatum has been implicated in perceptual decision making \parencite{Ding2010,Ding2013}, few studies have investigated corticostriatal circuits in the context of perceptual decision making. Previous work by \textcite{Znamenskiy2013b} demonstrated that corticostriatal neurons in auditory cortex can selective bias and drive behavioral decisions. It is not known whether this generalizes to striate or extrastriate visual cortices. The striatum, and other subcortical motor areas, is interesting because it predates the neocortex, is coupled to motor circuits in the brain stem, and receives sensory inputs from multiple brain areas. Hence it is likely an important hub for understanding how sensory and perceptual information is converted into action. \par

Further work is needed to understand the functional roles of mouse secondary visual areas and their involvement in perceptual decision making behavior. The approach used in this thesis, targeted a single area across multiple mice. Although this low-throughput approach could be scaled up by running large cohorts of animals in parallel, each with an inactivation target to a particular visual area, it is not ideal for understanding how multiple areas work in concert to generate behavior. Hence, future work could leverage large-scale tools for perturbing and/or recording neural activity from multiple brain areas simultaneously (or combinatorially). For example, the paradigm described in this thesis can be optical neural recording techniques such as head-mounded microscopes \parencite{Ghosh2011} or multi-area fiber photometry \parencite{Kim2016}. With the right configuration, the multi-area fiber photometry system could also be used for photoinhibition. Alternatively, an analogous head-restrained version of the task \parencite{Marbach2016} could be implemented and combined with patterned photoinhibition \parencite{Dhawale2010,Guo2014a}, widefield imaging \parencite{Wekselblatt2016}, and/or cellular-level mesoscale imaging \parencite{Stirman2016a,Sofroniew2016} in behaving mice. The possibilities are endless.

In summary, the underlying theme of this thesis was to understand how sensory and perceptual information is transformed into categorical decisions and ultimately actions. Perhaps for the first time in systems neuroscience, we are living in an age of abundant tools for probing, manipulating, and monitoring neural circuit components and at the center of the technological revolution is the mouse. With carefully designed behavioral paradigms and stimuli that harness evolved features of their sensory systems, mice are capable of performing complex behavioral task that require cortical machinery. Therefore, now is the time to leverage existing behavioral paradigms and tools to discover unknown computational algorithms implemented in biological tissue, and ultimately, inspire the design of the next generation of intelligent machines.


